\documentstyle[epsf,twoside,titlepage]{ugart}

% Voreinstellungen
%%%%%%%%%%%%%%%%%%%%%%%%%%%%%%%%%%%%%%%%%%%%%%%%%%%%%%%%%%%%%%%%%%%%%%%%%%%%%
%                             ug documentation                              %
%%%%%%%%%%%%%%%%%%%%%%%%%%%%%%%%%%%%%%%%%%%%%%%%%%%%%%%%%%%%%%%%%%%%%%%%%%%%%

% Sizes
\textwidth=155mm
\textheight=220mm
\headheight25pt
\headsep=12mm
\oddsidemargin=9mm
\evensidemargin=-4mm
% workstation:
\topmargin=-25mm
% Mac:
%\topmargin=0mm
\textfloatsep15mm
\parindent0mm
\setlength{\parskip}{6pt plus1pt minus2pt}
\leftmargin8mm
\renewcommand{\textfraction}{0.15}
\renewcommand{\topfraction}{0.9}
\renewcommand{\bottomfraction}{0.9}
\renewcommand{\floatpagefraction}{0.9}

%Rainer
\newcommand{\br}{15.5cm}                     % Textbreite f�r Boxen
\newcommand{\GRAD}{\, \mbox{\rm \bf grad} \,}        % sch�nes "grad"
\newcommand{\DIV}{\, \mbox{\rm div} \,}          % sch�nes "div"
\newcommand{\ds}[1]{\displaystyle{#1}}
\newcommand{\JAK}{\, {{\bf J}_{Jak}}}
\newcommand{\JI}{\, {{\bf J}_{Jak}^{-1}}}
\newcommand{\J}{\, \mbox{\rm det} \, {\bf J}_{Jak}}
\newcommand{\np}{n_{phas}}
\newcommand{\nel}{n_{elem}}
\newcommand{\nk}{n_{kno}}
\newcommand{\Sum}{\sum\limits}
\newcommand{\Int}{\int\limits}
\newcommand{\na}{\sc}
\newcommand{\au}{\sf}
\newcommand{\K}{{\, \bf K}}
\newcommand{\FAK}{F\!a\!k}



% Theorems
\newtheorem{algo}{\noindent\bf Algorithm}[section]
\newtheorem{remark}{\noindent\bf Remark}[section]
\newtheorem{definition}{\noindent\bf Definition}[section]
\newtheorem{lemma}{\noindent\bf Lemma}[section]
\newtheorem{unit}{\mbox{}}[section]

% Macros
\newcommand{\Abb}[4]{
   \begin{figure}[#1]
   \centerline{\epsffile{#2}}
   \caption{#4}
   \label{#3}
   \end{figure}
}
\newcommand{\Absatz}[1]{
   {\medskip\noindent\bf #1}
}
\newcommand{\DoubleBox}[2]{%
\begin{minipage}[t]{6cm} #1 \end{minipage}\hfill
\begin{minipage}[t]{9cm} #2 \end{minipage}
}
\newcommand{\headerfile}[1]{
   {\parindent10mm\bf #1}
}

% Abkuerzungen
\newcommand{\spann}{\mbox{span}}
\newcommand{\meas}{\mbox{meas}}
%\newcommand{\DIV}{\mbox{div}}
\newcommand{\TT}{{\cal T}}
\newcommand{\CC}{{\cal C}}
\newcommand{\SS}{{\cal S}}
\newcommand{\AAA}{{\cal A}}
\newcommand{\QQ}{{\cal Q}}
\newcommand{\RR}{{\cal R}}
\newcommand{\PP}{{\cal P}}
\newcommand{\VV}{{\cal V}}
\newcommand{\LL}{{\cal L}}
\newcommand{\BB}{{\cal B}}
\newcommand{\EE}{{\cal E}}
\newcommand{\II}{{\cal I}}
\newcommand{\OO}{{\cal O}}
\newcommand{\Real}{\mbox{\bf R}}
\newcommand{\Natural}{\mbox{\bf N}}
\newcommand{\Integer}{\mbox{\bf Z}}
\newcommand{\SP}{\hspace{5mm}}
\newcommand{\bigsp}{\hspace{5mm}}
\newcommand{\ssp}{\hspace{1em}}
\newcommand{\UG}{{\sl UG }}
\newcommand{\Prio}{\mbox{prio}}
\newcommand{\Mark}{\mbox{mark}}
\newcommand{\diag}{\mbox{diag}}
\newcommand{\blockdiag}{\mbox{blockdiag}}
\newcommand{\Bottom}{\mbox{bottom}}
\newcommand{\Top}{\mbox{top}}
\newcommand{\Root}{\mbox{root}}
\newcommand{\mod}{\mbox{mod}}
\newcommand{\Regular}{\mbox{regular}}
\newcommand{\Temporary}{\mbox{temporary}}
\newcommand{\Copy}{\mbox{copy}}
\newcommand{\Level}{\mbox{level}}

\frenchspacing

\newcommand{\mantitle}[3]{
\markboth{Formatted: #3}{#1(#2) \hfil \sectitle \hfil #1(#2)}
\addcontentsline{toc}{subsection}{#1}
\index{#1}
}

\newcommand{\manname}[1]{
{\bf NAME} 
\begin{list}{\mbox{}}{\topsep0cm \partopsep0cm \leftmargin1cm}
\item[\mbox{}] #1
}

\newcommand{\subhead}[1]{\nopagebreak
\end{list}
\vspace{4mm}
\pagebreak[2]
{\bf #1} 
\begin{list}{\mbox{}}{\topsep0cm \partopsep0cm \leftmargin1cm}
\item[\mbox{}] 
%\mbox{}
}

\newcommand{\startarg}[2]{
\vspace{3mm}
{\em #1} 
\begin{list}{\mbox{}}{\topsep0cm \partopsep0cm \leftmargin1cm}
\item[\mbox{}] 
\mbox{} 
#2
\end{list}
}

\newcommand{\startvb}{ \vspace{0mm} }

\newcommand{\myendvb}{ \mbox{} }

\newcommand{\nextline}{ \par \hspace*{1cm} }

\newcommand{\location}[1]{
\end{list}
\vspace{4mm}
{\bf LOCATION} 
\begin{list}{\mbox{}}{\topsep0cm \partopsep0cm \leftmargin1cm}
\item[\mbox{}] #1
}

\newcommand{\myendmanpage}{%
\end{list}
\vfill
\clearpage
}


\pagestyle{myheadings}
\sloppy
\makeindex

% Titelseite


\begin{document}
\newcommand{\sectitle}{\mbox{}}
\setcounter{page}{0}

\title{UG Version 3.1 Reference Manual}
\author{The UG Group}
\maketitle



\section*{Introduction}

The UG Manual consists of three parts:
\\[5mm]  
The {\em Programming Manual} gives an overview on the program, it
explains the submodules and all data structures and it contains
all commands which are needed for running an application.
\\[5mm]  
The {\em Reference Manual} contains a list of all 
functions in the ug library which can be called in the Problem Classes.
\\[5mm]
The {\em Application and Problem Class Documentation} gives examples
of applications of the ug library.
\\[5mm]
The manuals are created with the tool {\tt doctext} and the content is 
identical with the man pages. All man pages can be displayed on screen 
with the commands {\tt xugman} resp. {\tt ugman}.
\\[5mm]
We hope that this documentation helps the reader to use UG 
and to implement own applications.
\\[1cm]
The documentation team
\\[1cm]
Peter Bastian\\
Gabi Beddies\\
Rolf Dornberger\\
Dirk Feuchter\\
Reinhard Haag\\
Ingo Heppner\\
Wolfgang Hoffmann\\
Klaus Johannsen\\
Stefan Lang\\
George Mazurkevich\\
Nikolas Neu\ss\\
Henrik Rentz-Reichert\\
Oliver Urbitsch\\
Christian Wieners\\
Christian Wrobel

\newpage
{
\small
\tableofcontents
}
\clearpage

% include the different sections
\thispagestyle{plain}
\section{manpages for module ''ug''}
\renewcommand{\sectitle}{ug}
\newcommand{\mantitle}[3]{
\markboth{Formatted: #3}{#1(#2) \hfil \sectitle \hfil #1(#2)}
\addcontentsline{toc}{subsection}{#1}
\index{#1}
}

\newcommand{\manname}[1]{
{\bf NAME} 
\begin{list}{\mbox{}}{\topsep0cm \partopsep0cm \leftmargin1cm}
\item[\mbox{}] #1
}

\newcommand{\subhead}[1]{\nopagebreak
\end{list}
\vspace{4mm}
\pagebreak[2]
{\bf #1} 
\begin{list}{\mbox{}}{\topsep0cm \partopsep0cm \leftmargin1cm}
\item[\mbox{}] 
%\mbox{}
}

\newcommand{\startarg}[2]{
\vspace{3mm}
{\em #1} 
\begin{list}{\mbox{}}{\topsep0cm \partopsep0cm \leftmargin1cm}
\item[\mbox{}] 
\mbox{} 
#2
\end{list}
}

\newcommand{\startvb}{ \vspace{0mm} }

\newcommand{\myendvb}{ \mbox{} }

\newcommand{\nextline}{ \par \hspace*{1cm} }

\newcommand{\location}[1]{
\end{list}
\vspace{4mm}
{\bf LOCATION} 
\begin{list}{\mbox{}}{\topsep0cm \partopsep0cm \leftmargin1cm}
\item[\mbox{}] #1
}

\newcommand{\myendmanpage}{%
\end{list}
\vfill
\clearpage
}



\thispagestyle{plain}
\section{manpages for module ''ug/dev''}
\renewcommand{\sectitle}{ug/dev}
\newcommand{\mantitle}[3]{
\markboth{Formatted: #3}{#1(#2) \hfil \sectitle \hfil #1(#2)}
\addcontentsline{toc}{subsection}{#1}
\index{#1}
}

\newcommand{\manname}[1]{
{\bf NAME} 
\begin{list}{\mbox{}}{\topsep0cm \partopsep0cm \leftmargin1cm}
\item[\mbox{}] #1
}

\newcommand{\subhead}[1]{\nopagebreak
\end{list}
\vspace{4mm}
\pagebreak[2]
{\bf #1} 
\begin{list}{\mbox{}}{\topsep0cm \partopsep0cm \leftmargin1cm}
\item[\mbox{}] 
%\mbox{}
}

\newcommand{\startarg}[2]{
\vspace{3mm}
{\em #1} 
\begin{list}{\mbox{}}{\topsep0cm \partopsep0cm \leftmargin1cm}
\item[\mbox{}] 
\mbox{} 
#2
\end{list}
}

\newcommand{\startvb}{ \vspace{0mm} }

\newcommand{\myendvb}{ \mbox{} }

\newcommand{\nextline}{ \par \hspace*{1cm} }

\newcommand{\location}[1]{
\end{list}
\vspace{4mm}
{\bf LOCATION} 
\begin{list}{\mbox{}}{\topsep0cm \partopsep0cm \leftmargin1cm}
\item[\mbox{}] #1
}

\newcommand{\myendmanpage}{%
\end{list}
\vfill
\clearpage
}



% shall not be documanted!
%\thispagestyle{plain}
%\section{manpages for module ''ug/dev/meta''}
%\renewcommand{\sectitle}{ug/dev/meta}
%\newcommand{\mantitle}[3]{
\markboth{Formatted: #3}{#1(#2) \hfil \sectitle \hfil #1(#2)}
\addcontentsline{toc}{subsection}{#1}
\index{#1}
}

\newcommand{\manname}[1]{
{\bf NAME} 
\begin{list}{\mbox{}}{\topsep0cm \partopsep0cm \leftmargin1cm}
\item[\mbox{}] #1
}

\newcommand{\subhead}[1]{\nopagebreak
\end{list}
\vspace{4mm}
\pagebreak[2]
{\bf #1} 
\begin{list}{\mbox{}}{\topsep0cm \partopsep0cm \leftmargin1cm}
\item[\mbox{}] 
%\mbox{}
}

\newcommand{\startarg}[2]{
\vspace{3mm}
{\em #1} 
\begin{list}{\mbox{}}{\topsep0cm \partopsep0cm \leftmargin1cm}
\item[\mbox{}] 
\mbox{} 
#2
\end{list}
}

\newcommand{\startvb}{ \vspace{0mm} }

\newcommand{\myendvb}{ \mbox{} }

\newcommand{\nextline}{ \par \hspace*{1cm} }

\newcommand{\location}[1]{
\end{list}
\vspace{4mm}
{\bf LOCATION} 
\begin{list}{\mbox{}}{\topsep0cm \partopsep0cm \leftmargin1cm}
\item[\mbox{}] #1
}

\newcommand{\myendmanpage}{%
\end{list}
\vfill
\clearpage
}



% shall not be documanted!
%\thispagestyle{plain}
%\section{manpages for module ''ug/dev/mif''}
%\renewcommand{\sectitle}{ug/dev/mif}
%\newcommand{\mantitle}[3]{
\markboth{Formatted: #3}{#1(#2) \hfil \sectitle \hfil #1(#2)}
\addcontentsline{toc}{subsection}{#1}
\index{#1}
}

\newcommand{\manname}[1]{
{\bf NAME} 
\begin{list}{\mbox{}}{\topsep0cm \partopsep0cm \leftmargin1cm}
\item[\mbox{}] #1
}

\newcommand{\subhead}[1]{\nopagebreak
\end{list}
\vspace{4mm}
\pagebreak[2]
{\bf #1} 
\begin{list}{\mbox{}}{\topsep0cm \partopsep0cm \leftmargin1cm}
\item[\mbox{}] 
%\mbox{}
}

\newcommand{\startarg}[2]{
\vspace{3mm}
{\em #1} 
\begin{list}{\mbox{}}{\topsep0cm \partopsep0cm \leftmargin1cm}
\item[\mbox{}] 
\mbox{} 
#2
\end{list}
}

\newcommand{\startvb}{ \vspace{0mm} }

\newcommand{\myendvb}{ \mbox{} }

\newcommand{\nextline}{ \par \hspace*{1cm} }

\newcommand{\location}[1]{
\end{list}
\vspace{4mm}
{\bf LOCATION} 
\begin{list}{\mbox{}}{\topsep0cm \partopsep0cm \leftmargin1cm}
\item[\mbox{}] #1
}

\newcommand{\myendmanpage}{%
\end{list}
\vfill
\clearpage
}



% shall not be documanted!
%\thispagestyle{plain}
%\section{manpages for module ''ug/dev/xif''}
%\renewcommand{\sectitle}{ug/dev/xif}
%\newcommand{\mantitle}[3]{
\markboth{Formatted: #3}{#1(#2) \hfil \sectitle \hfil #1(#2)}
\addcontentsline{toc}{subsection}{#1}
\index{#1}
}

\newcommand{\manname}[1]{
{\bf NAME} 
\begin{list}{\mbox{}}{\topsep0cm \partopsep0cm \leftmargin1cm}
\item[\mbox{}] #1
}

\newcommand{\subhead}[1]{\nopagebreak
\end{list}
\vspace{4mm}
\pagebreak[2]
{\bf #1} 
\begin{list}{\mbox{}}{\topsep0cm \partopsep0cm \leftmargin1cm}
\item[\mbox{}] 
%\mbox{}
}

\newcommand{\startarg}[2]{
\vspace{3mm}
{\em #1} 
\begin{list}{\mbox{}}{\topsep0cm \partopsep0cm \leftmargin1cm}
\item[\mbox{}] 
\mbox{} 
#2
\end{list}
}

\newcommand{\startvb}{ \vspace{0mm} }

\newcommand{\myendvb}{ \mbox{} }

\newcommand{\nextline}{ \par \hspace*{1cm} }

\newcommand{\location}[1]{
\end{list}
\vspace{4mm}
{\bf LOCATION} 
\begin{list}{\mbox{}}{\topsep0cm \partopsep0cm \leftmargin1cm}
\item[\mbox{}] #1
}

\newcommand{\myendmanpage}{%
\end{list}
\vfill
\clearpage
}



% no manpages
%\thispagestyle{plain}
%\section{manpages for module ''ug/dom''}
%\renewcommand{\sectitle}{ug/dom}
%\newcommand{\mantitle}[3]{
\markboth{Formatted: #3}{#1(#2) \hfil \sectitle \hfil #1(#2)}
\addcontentsline{toc}{subsection}{#1}
\index{#1}
}

\newcommand{\manname}[1]{
{\bf NAME} 
\begin{list}{\mbox{}}{\topsep0cm \partopsep0cm \leftmargin1cm}
\item[\mbox{}] #1
}

\newcommand{\subhead}[1]{\nopagebreak
\end{list}
\vspace{4mm}
\pagebreak[2]
{\bf #1} 
\begin{list}{\mbox{}}{\topsep0cm \partopsep0cm \leftmargin1cm}
\item[\mbox{}] 
%\mbox{}
}

\newcommand{\startarg}[2]{
\vspace{3mm}
{\em #1} 
\begin{list}{\mbox{}}{\topsep0cm \partopsep0cm \leftmargin1cm}
\item[\mbox{}] 
\mbox{} 
#2
\end{list}
}

\newcommand{\startvb}{ \vspace{0mm} }

\newcommand{\myendvb}{ \mbox{} }

\newcommand{\nextline}{ \par \hspace*{1cm} }

\newcommand{\location}[1]{
\end{list}
\vspace{4mm}
{\bf LOCATION} 
\begin{list}{\mbox{}}{\topsep0cm \partopsep0cm \leftmargin1cm}
\item[\mbox{}] #1
}

\newcommand{\myendmanpage}{%
\end{list}
\vfill
\clearpage
}



\thispagestyle{plain}
\section{manpages for module ''ug/dom/std''}
\renewcommand{\sectitle}{ug/dom/std}
\newcommand{\mantitle}[3]{
\markboth{Formatted: #3}{#1(#2) \hfil \sectitle \hfil #1(#2)}
\addcontentsline{toc}{subsection}{#1}
\index{#1}
}

\newcommand{\manname}[1]{
{\bf NAME} 
\begin{list}{\mbox{}}{\topsep0cm \partopsep0cm \leftmargin1cm}
\item[\mbox{}] #1
}

\newcommand{\subhead}[1]{\nopagebreak
\end{list}
\vspace{4mm}
\pagebreak[2]
{\bf #1} 
\begin{list}{\mbox{}}{\topsep0cm \partopsep0cm \leftmargin1cm}
\item[\mbox{}] 
%\mbox{}
}

\newcommand{\startarg}[2]{
\vspace{3mm}
{\em #1} 
\begin{list}{\mbox{}}{\topsep0cm \partopsep0cm \leftmargin1cm}
\item[\mbox{}] 
\mbox{} 
#2
\end{list}
}

\newcommand{\startvb}{ \vspace{0mm} }

\newcommand{\myendvb}{ \mbox{} }

\newcommand{\nextline}{ \par \hspace*{1cm} }

\newcommand{\location}[1]{
\end{list}
\vspace{4mm}
{\bf LOCATION} 
\begin{list}{\mbox{}}{\topsep0cm \partopsep0cm \leftmargin1cm}
\item[\mbox{}] #1
}

\newcommand{\myendmanpage}{%
\end{list}
\vfill
\clearpage
}



\thispagestyle{plain}
\section{manpages for module ''ug/gm''}
\renewcommand{\sectitle}{ug/gm}
\newcommand{\mantitle}[3]{
\markboth{Formatted: #3}{#1(#2) \hfil \sectitle \hfil #1(#2)}
\addcontentsline{toc}{subsection}{#1}
\index{#1}
}

\newcommand{\manname}[1]{
{\bf NAME} 
\begin{list}{\mbox{}}{\topsep0cm \partopsep0cm \leftmargin1cm}
\item[\mbox{}] #1
}

\newcommand{\subhead}[1]{\nopagebreak
\end{list}
\vspace{4mm}
\pagebreak[2]
{\bf #1} 
\begin{list}{\mbox{}}{\topsep0cm \partopsep0cm \leftmargin1cm}
\item[\mbox{}] 
%\mbox{}
}

\newcommand{\startarg}[2]{
\vspace{3mm}
{\em #1} 
\begin{list}{\mbox{}}{\topsep0cm \partopsep0cm \leftmargin1cm}
\item[\mbox{}] 
\mbox{} 
#2
\end{list}
}

\newcommand{\startvb}{ \vspace{0mm} }

\newcommand{\myendvb}{ \mbox{} }

\newcommand{\nextline}{ \par \hspace*{1cm} }

\newcommand{\location}[1]{
\end{list}
\vspace{4mm}
{\bf LOCATION} 
\begin{list}{\mbox{}}{\topsep0cm \partopsep0cm \leftmargin1cm}
\item[\mbox{}] #1
}

\newcommand{\myendmanpage}{%
\end{list}
\vfill
\clearpage
}



\thispagestyle{plain}
\section{manpages for module ''ug/gm/gg2''}
\renewcommand{\sectitle}{ug/gm/gg2}
\newcommand{\mantitle}[3]{
\markboth{Formatted: #3}{#1(#2) \hfil \sectitle \hfil #1(#2)}
\addcontentsline{toc}{subsection}{#1}
\index{#1}
}

\newcommand{\manname}[1]{
{\bf NAME} 
\begin{list}{\mbox{}}{\topsep0cm \partopsep0cm \leftmargin1cm}
\item[\mbox{}] #1
}

\newcommand{\subhead}[1]{\nopagebreak
\end{list}
\vspace{4mm}
\pagebreak[2]
{\bf #1} 
\begin{list}{\mbox{}}{\topsep0cm \partopsep0cm \leftmargin1cm}
\item[\mbox{}] 
%\mbox{}
}

\newcommand{\startarg}[2]{
\vspace{3mm}
{\em #1} 
\begin{list}{\mbox{}}{\topsep0cm \partopsep0cm \leftmargin1cm}
\item[\mbox{}] 
\mbox{} 
#2
\end{list}
}

\newcommand{\startvb}{ \vspace{0mm} }

\newcommand{\myendvb}{ \mbox{} }

\newcommand{\nextline}{ \par \hspace*{1cm} }

\newcommand{\location}[1]{
\end{list}
\vspace{4mm}
{\bf LOCATION} 
\begin{list}{\mbox{}}{\topsep0cm \partopsep0cm \leftmargin1cm}
\item[\mbox{}] #1
}

\newcommand{\myendmanpage}{%
\end{list}
\vfill
\clearpage
}



\thispagestyle{plain}
\section{manpages for module ''ug/gm/gg3''}
\renewcommand{\sectitle}{ug/gm/gg3}
\newcommand{\mantitle}[3]{
\markboth{Formatted: #3}{#1(#2) \hfil \sectitle \hfil #1(#2)}
\addcontentsline{toc}{subsection}{#1}
\index{#1}
}

\newcommand{\manname}[1]{
{\bf NAME} 
\begin{list}{\mbox{}}{\topsep0cm \partopsep0cm \leftmargin1cm}
\item[\mbox{}] #1
}

\newcommand{\subhead}[1]{\nopagebreak
\end{list}
\vspace{4mm}
\pagebreak[2]
{\bf #1} 
\begin{list}{\mbox{}}{\topsep0cm \partopsep0cm \leftmargin1cm}
\item[\mbox{}] 
%\mbox{}
}

\newcommand{\startarg}[2]{
\vspace{3mm}
{\em #1} 
\begin{list}{\mbox{}}{\topsep0cm \partopsep0cm \leftmargin1cm}
\item[\mbox{}] 
\mbox{} 
#2
\end{list}
}

\newcommand{\startvb}{ \vspace{0mm} }

\newcommand{\myendvb}{ \mbox{} }

\newcommand{\nextline}{ \par \hspace*{1cm} }

\newcommand{\location}[1]{
\end{list}
\vspace{4mm}
{\bf LOCATION} 
\begin{list}{\mbox{}}{\topsep0cm \partopsep0cm \leftmargin1cm}
\item[\mbox{}] #1
}

\newcommand{\myendmanpage}{%
\end{list}
\vfill
\clearpage
}



\thispagestyle{plain}
\section{manpages for module ''ug/graphics''}
\renewcommand{\sectitle}{ug/graphics/uggraph}
\newcommand{\mantitle}[3]{
\markboth{Formatted: #3}{#1(#2) \hfil \sectitle \hfil #1(#2)}
\addcontentsline{toc}{subsection}{#1}
\index{#1}
}

\newcommand{\manname}[1]{
{\bf NAME} 
\begin{list}{\mbox{}}{\topsep0cm \partopsep0cm \leftmargin1cm}
\item[\mbox{}] #1
}

\newcommand{\subhead}[1]{\nopagebreak
\end{list}
\vspace{4mm}
\pagebreak[2]
{\bf #1} 
\begin{list}{\mbox{}}{\topsep0cm \partopsep0cm \leftmargin1cm}
\item[\mbox{}] 
%\mbox{}
}

\newcommand{\startarg}[2]{
\vspace{3mm}
{\em #1} 
\begin{list}{\mbox{}}{\topsep0cm \partopsep0cm \leftmargin1cm}
\item[\mbox{}] 
\mbox{} 
#2
\end{list}
}

\newcommand{\startvb}{ \vspace{0mm} }

\newcommand{\myendvb}{ \mbox{} }

\newcommand{\nextline}{ \par \hspace*{1cm} }

\newcommand{\location}[1]{
\end{list}
\vspace{4mm}
{\bf LOCATION} 
\begin{list}{\mbox{}}{\topsep0cm \partopsep0cm \leftmargin1cm}
\item[\mbox{}] #1
}

\newcommand{\myendmanpage}{%
\end{list}
\vfill
\clearpage
}



\thispagestyle{plain}
\section{manpages for module ''ug/low''}
\renewcommand{\sectitle}{ug/low}
\newcommand{\mantitle}[3]{
\markboth{Formatted: #3}{#1(#2) \hfil \sectitle \hfil #1(#2)}
\addcontentsline{toc}{subsection}{#1}
\index{#1}
}

\newcommand{\manname}[1]{
{\bf NAME} 
\begin{list}{\mbox{}}{\topsep0cm \partopsep0cm \leftmargin1cm}
\item[\mbox{}] #1
}

\newcommand{\subhead}[1]{\nopagebreak
\end{list}
\vspace{4mm}
\pagebreak[2]
{\bf #1} 
\begin{list}{\mbox{}}{\topsep0cm \partopsep0cm \leftmargin1cm}
\item[\mbox{}] 
%\mbox{}
}

\newcommand{\startarg}[2]{
\vspace{3mm}
{\em #1} 
\begin{list}{\mbox{}}{\topsep0cm \partopsep0cm \leftmargin1cm}
\item[\mbox{}] 
\mbox{} 
#2
\end{list}
}

\newcommand{\startvb}{ \vspace{0mm} }

\newcommand{\myendvb}{ \mbox{} }

\newcommand{\nextline}{ \par \hspace*{1cm} }

\newcommand{\location}[1]{
\end{list}
\vspace{4mm}
{\bf LOCATION} 
\begin{list}{\mbox{}}{\topsep0cm \partopsep0cm \leftmargin1cm}
\item[\mbox{}] #1
}

\newcommand{\myendmanpage}{%
\end{list}
\vfill
\clearpage
}



\thispagestyle{plain}
\section{manpages for module ''ug/numerics''}
\renewcommand{\sectitle}{ug/numerics}
\newcommand{\mantitle}[3]{
\markboth{Formatted: #3}{#1(#2) \hfil \sectitle \hfil #1(#2)}
\addcontentsline{toc}{subsection}{#1}
\index{#1}
}

\newcommand{\manname}[1]{
{\bf NAME} 
\begin{list}{\mbox{}}{\topsep0cm \partopsep0cm \leftmargin1cm}
\item[\mbox{}] #1
}

\newcommand{\subhead}[1]{\nopagebreak
\end{list}
\vspace{4mm}
\pagebreak[2]
{\bf #1} 
\begin{list}{\mbox{}}{\topsep0cm \partopsep0cm \leftmargin1cm}
\item[\mbox{}] 
%\mbox{}
}

\newcommand{\startarg}[2]{
\vspace{3mm}
{\em #1} 
\begin{list}{\mbox{}}{\topsep0cm \partopsep0cm \leftmargin1cm}
\item[\mbox{}] 
\mbox{} 
#2
\end{list}
}

\newcommand{\startvb}{ \vspace{0mm} }

\newcommand{\myendvb}{ \mbox{} }

\newcommand{\nextline}{ \par \hspace*{1cm} }

\newcommand{\location}[1]{
\end{list}
\vspace{4mm}
{\bf LOCATION} 
\begin{list}{\mbox{}}{\topsep0cm \partopsep0cm \leftmargin1cm}
\item[\mbox{}] #1
}

\newcommand{\myendmanpage}{%
\end{list}
\vfill
\clearpage
}



\thispagestyle{plain}
\section{manpages for module ''ug/ui''}
\renewcommand{\sectitle}{ug/ui}
\newcommand{\mantitle}[3]{
\markboth{Formatted: #3}{#1(#2) \hfil \sectitle \hfil #1(#2)}
\addcontentsline{toc}{subsection}{#1}
\index{#1}
}

\newcommand{\manname}[1]{
{\bf NAME} 
\begin{list}{\mbox{}}{\topsep0cm \partopsep0cm \leftmargin1cm}
\item[\mbox{}] #1
}

\newcommand{\subhead}[1]{\nopagebreak
\end{list}
\vspace{4mm}
\pagebreak[2]
{\bf #1} 
\begin{list}{\mbox{}}{\topsep0cm \partopsep0cm \leftmargin1cm}
\item[\mbox{}] 
%\mbox{}
}

\newcommand{\startarg}[2]{
\vspace{3mm}
{\em #1} 
\begin{list}{\mbox{}}{\topsep0cm \partopsep0cm \leftmargin1cm}
\item[\mbox{}] 
\mbox{} 
#2
\end{list}
}

\newcommand{\startvb}{ \vspace{0mm} }

\newcommand{\myendvb}{ \mbox{} }

\newcommand{\nextline}{ \par \hspace*{1cm} }

\newcommand{\location}[1]{
\end{list}
\vspace{4mm}
{\bf LOCATION} 
\begin{list}{\mbox{}}{\topsep0cm \partopsep0cm \leftmargin1cm}
\item[\mbox{}] #1
}

\newcommand{\myendmanpage}{%
\end{list}
\vfill
\clearpage
}




\end{document}

